\documentclass[11pt,oneside]{article}

\usepackage{amsfonts,amsmath,amsthm,amssymb}
\usepackage{hyperref}
\usepackage[utf8]{inputenc}

%%%%
%\usepackage{fancyhdr}
%\pagestyle{fancy}
%\renewcommand{\headrulewidth}{0pt}
%\fancyhf{}

%\setlength{\textheight}{9.6in}
%\setlength{\textwidth}{7.0in}
%\setlength{\topmargin}{-0.8in}
%\setlength{\oddsidemargin}{-0.25in}
%\setlength{\evensidemargin}{0.75in}
%\setlength{\parskip}{0.15in}
%\setlength{\parindent}{0in}
%%%

\title{MATH 135: Extra Practice Set 5}
\author{imjching}

\newtheorem{question}{Question}

\begin{document}

\begin{center}
    \textbf{MATH 135: Extra Practice Set 5} \\
    December 21$^{st}$ 2016 \\

    imjching
\end{center}

\bigskip

% Recommended Problems

% Question 1
\begin{question}
	$(a)$ Use the Extended Euclidean Algorithm to find three integers $x$, $y$ and $d = gcd(1112, 768)$ such that $1112x + 768y = d$. $(b)$ Determine integers $s$ and $t$ such that $768s - 1112t = gcd(768, -1112)$.
\end{question}

\bigskip

% Question 2
\begin{question}
	Prove that for all a $\in \mathbb{Z}$, $gcd(9a + 4, 2a + 1) = 1$.
\end{question}

\bigskip

% Question 3
\begin{question}
	Let $gcd(x, y) = d$. Express $gcd(18x + 3 y, 3x)$ in terms of $d$ and prove that you are correct.
\end{question}

\bigskip

% Question 4
\begin{question}
	Prove that if $gcd(a, b) = 1$, then $gcd(2a + b, a + 2b) \in \{1, 3\}$.
\end{question}

\bigskip

% Question 5
\begin{question}
	Prove that for every integer $k$, $gcd(a, b) \leq gcd(ak, b)$.
\end{question}

\bigskip

% Question 6
\begin{question}
	Given a rational number $r$, prove that there exist coprime integers $p$ and $q$, with $q \neq 0$, so that $r = \frac{p}{q}$.
\end{question}

\bigskip

% Question 7
\begin{question}
	Prove that: if $a \mid c$ and $b \mid c$ and $gcd(a, b) = 1$, then $ab \mid c$.
\end{question}

\bigskip

% Question 8
\begin{question}
	Let $a, b, c \in \mathbb{Z}$. Prove that if $gcd(a, b) = 1$ and $c \mid a$, then $gcd(b, c) = 1$.
\end{question}

\bigskip

% Question 9
\begin{question}
	Prove that if $gcd(a, b) = 1$, then $gcd(a^{m}, b^{n}) = 1$ for all $m, n \in \mathbb{N}$. You may use the result of an example in the notes.
\end{question}

\bigskip

% Question 10
\begin{question}
	Suppose $a, b$ and $n$ are integers. Prove that $n \mid gcd(a, n) \cdot gcd(b, n)$ if and only if $n \mid ab$.
\end{question}

\end{document}
