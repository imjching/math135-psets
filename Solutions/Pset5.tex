\documentclass[11pt,oneside]{article}

\usepackage{amsfonts,amsmath,amsthm,amssymb}
\usepackage{hyperref}
\usepackage[utf8]{inputenc}

%%%%
%\usepackage{fancyhdr}
%\pagestyle{fancy}
%\renewcommand{\headrulewidth}{0pt}
%\fancyhf{}

%\setlength{\textheight}{9.6in}
%\setlength{\textwidth}{7.0in}
%\setlength{\topmargin}{-0.8in}
%\setlength{\oddsidemargin}{-0.25in}
%\setlength{\evensidemargin}{0.75in}
%\setlength{\parskip}{0.15in}
%\setlength{\parindent}{0in}
%%%

\title{MATH 135: Extra Practice Set 5}
\author{Jay Ching Lim}

\newtheorem{question}{Question}

\begin{document}

\begin{center}
    \textbf{MATH 135: Extra Practice Set 5} \\
    January 11$^{th}$ 2017 \\

    Jay Ching Lim
\end{center}

\bigskip

% Recommended Problems

% Question 1
\begin{question}
	$(a)$ Use the Extended Euclidean Algorithm to find three integers $x$, $y$ and $d = \text{gcd}(1112, 768)$ such that $1112x + 768y = d$. $(b)$ Determine integers $s$ and $t$ such that $768s - 1112t = \text{gcd}(768, -1112)$.
\end{question}

\begin{enumerate}
	\item[(a)]  We have $1112x + 768y = \text{gcd}(1112, 768) = d$. By EEA,

\begin{center}
	\begin{tabular}{c|c|c|c}
		x & y & r & q \\
		\hline
		1 & 0 & 1112 & 0\\
		0 & 1 & 768 & 0\\
		1 & -1 & 344 & 1\\
		-2 & 3 & 80 & 2\\
		9 & -13 & 24 & 4\\
		-29 & 42 & 8 & 3\\
		96 & -139 & 0 & 3\\
	\end{tabular}
\end{center}

From the second last row, we have $$1112(-29) + 768(42) = \text{gcd}(1112, 768) = 8.$$ Thus, $x = -29$, $y= 42$, and $d = 8$.

	\item[(b)] Observe that $\text{gcd}(768, -1112) = \text{gcd}(-1112, 768) = \text{gcd}(1112, 768)$.
	
Let $s = y$ and $t = -x$. Then we need to determine integers $s$ and $t$ such that $768y + 1112x = \text{gcd}(1112, 768)$, which is the same equation as before. Since we have $1112(-29) + 768(42) = \text{gcd}(1112, 768) = 8$, we can deduce that $s = 42$ and $t = -(-29) = 29$.

\end{enumerate}

% Question 2
\begin{question}
	Prove that for all a $\in \mathbb{Z}$, $\text{gcd}(9a + 4, 2a + 1) = 1$.
\end{question}

\begin{proof}
	Let $a \in \mathbb{Z}$. We will apply GCD WR repeatedly.\\
	
	Since $9a+4$ = $4(2a+1)+a$, $\text{gcd}(9a+4,2a+1)=\text{gcd}(2a+1,a)$.\\
	Since $2a+1$ = $2(a)+1$, $\text{gcd}(2a+1,a)=\text{gcd}(a,1)$.\\
	Since $a$ = $a(1)+0$, $\text{gcd}(a,1)=\text{gcd}(1,0)=\vert1\vert=1$.\\
	
	By the chain of equalities, we get
	\begin{align*}
		\text{gcd}(9a+4,2a+1) &= \text{gcd}(2a+1,a)\\
		&=\text{gcd}(a,1)\\
		&=\text{gcd}(1,0)\\
		&=\vert1\vert\\
		&=1\text{, as required.}
	\end{align*}
\end{proof}

\bigskip

% Question 3
\begin{question}
	Let $\text{gcd}(x, y) = d$. Express $\text{gcd}(18x + 3 y, 3x)$ in terms of $d$ and prove that you are correct.
\end{question}

\bigskip

We know that $18x + 3y=6(3x) + 3y$. So by GCD WR, $\text{gcd}(18x + 3y,3x)=\text{gcd}(3x,3y)=3\cdot\text{gcd}(x,y)=3d$.

\begin{proof}
	Since $d=gcd(x,y)$, $d | x$ and $d | y$ by the definition of gcd. 
	By the definition of divisibility, $\exists k\in\mathbb{Z}$ such that $dk = x$.
	Multiplying this by $3$, we get $(3d)k=3x$. Since $k \in \mathbb{Z}$, $3d \vert 3k$. 
	Also, $\exists h\in\mathbb{Z}$ such that $dh = y$ and a similar argument shows that $3d|3y$.\\\\
	By BL, $\exists x_1,y_1\in\mathbb{Z}$ such that $$xx_1 + yy_1 = d.$$ Multiplying the equation by $3$ yields $$(3x)x_1 + (3y)y_1 = 3d.$$
	Using our previous results, i.e. $3d|3x$ and $3d|3y$ together with the fact that $\exists x_1,y_1\in\mathbb{Z}$ such that $(3x)x_1 + (3y)y_1 = 3d$, we can apply GCD CT to deduce that $\text{gcd}(18x + 3y,3x)=3d$, as required.\\
\end{proof}

\clearpage

% Question 4
\begin{question}
	Prove that if $gcd(a, b) = 1$, then $gcd(2a + b, a + 2b) \in \{1, 3\}$.
\end{question}

\begin{proof}
	Assume that $gcd(a, b) = 1$.
	Let $d=gcd(2a + b, a + 2b)$. By the definition of gcd, $d|(2a+b)$ and $d|(a+2b)$.
	By DIC, $$d|[2(2a+b) + (-1)(a+2b)] \implies d | 3a.$$
	Again, by DIC, $$d|[(-1)(2a+b) + 2(a+2b)] \implies d | 3b.$$
	By GCD OO, $\exists x,y\in\mathbb{Z}$ such that $ax+by=1$. Multiplying this equation by 3, we get $(3a)x + (3b)y = 3$. By the definition of divisibility, $\exists k\in\mathbb{Z}$ such that $dk = 3a$ and $\exists h\in\mathbb{Z}$ such that $dh = 3b$. Substituting $3a = dk$ and $3b = dh$ into the previous equation, we get $(dk)x + (dh)y = 3 \iff d(kx+hy) = 3$. Since $k,x,h,y \in \mathbb{Z}$, $kx+hy \in \mathbb{Z}$. So by definition of divisibility, $d|3$. The only possible values of $d$ are $1$ and $3$, i.e. $d = gcd(2a + b, a + 2b) \in \{1, 3\}$, as required.\\
\end{proof}

% Question 5
\begin{question}
	Prove that for every integer $k$, $gcd(a, b) \leq gcd(ak, b)$.
\end{question}

\begin{proof}
	Let $k\in\mathbb{Z}$, $d_1=gcd(a, b)$ and $d_2=gcd(ak, b)$. By definition of gcd, $d_1|a$ and $d_1|b$. Clearly, $a|ak$. By TD, since $d_1|a$ and $a|ak$, $d_1|ak$. Applying the definition of gcd on $d_2$, we know that $\forall c\in\mathbb{Z}$, if $c|ak$ and $c|b$, then $c \leq gcd(ak, b) = d_2$. Since $d_1|ak$ and $d_1|b$, we can let $c = d_1$ to deduce that $d_1 \leq d_2$. Thus, $gcd(a, b) \leq gcd(ak, b)$, as required.\\
\end{proof}

% Question 6
\begin{question}
	Given a rational number $r$, prove that there exist coprime integers $p$ and $q$, with $q \neq 0$, so that $r = \frac{p}{q}$.
\end{question}

\bigskip

% Question 7
\begin{question}
	Prove that: if $a \mid c$ and $b \mid c$ and $gcd(a, b) = 1$, then $ab \mid c$.
\end{question}

\bigskip

% Question 8
\begin{question}
	Let $a, b, c \in \mathbb{Z}$. Prove that if $gcd(a, b) = 1$ and $c \mid a$, then $gcd(b, c) = 1$.
\end{question}

\bigskip

% Question 9
\begin{question}
	Prove that if $gcd(a, b) = 1$, then $gcd(a^{m}, b^{n}) = 1$ for all $m, n \in \mathbb{N}$. You may use the result of an example in the notes.
\end{question}

\bigskip

% Question 10
\begin{question}
	Suppose $a, b$ and $n$ are integers. Prove that $n \mid gcd(a, n) \cdot gcd(b, n)$ if and only if $n \mid ab$.
\end{question}

\end{document}
